%!TEX TS-program = latex
\documentclass[epsfig,12pt]{article}

\input PH631_layout2
\begin{document}

\input extra_macros
\input PH631_logo\setcounter{part}{5}



\bigskip

{\LARGE \bf Conductors and the Electrostatic Potentlal}

\begin{tcolorbox}
We're going to do Jackson sections 1.1-1.5 and then do 1.11 on energy and capacitance before looping back to boundary conditions and Greens' functions (1.7-1.10) after we solve some potential problems using the method of images. 
\end{tcolorbox}

\section{Conductors}

So far we have only used fixed charge distributions.
Now we will introduce the 'ideal' conductor which has the property that
the charges within it can move freely under the influence of electrical forces.

Say we take a chunk of conducting material and immerse it in a uniform
electric field?  All of the free electrons will experience a force and move
until there is no more force on them.  That force will be $q\vE_{\rm TOT}$ where
the total field is the sum of the external field and fields due to the electrons
and protons themselves. The electrons will move so that their field cancels
the external field.  If $F= 0 $ inside the conductor, $\vE_{\rm TOT}$ must be
as well once the charges stop moving.
\bigskip

\centerline{AT EQUILIBRIUM $\vE = 0$ INSIDE A PERFECT CONDUCTOR}
\bigskip

\subsection{Fields near the surface of the conductor}

As the field is identically zero inside the conductor and the field has to 
satisfy the condition $\Delta \vE_{\rm tot} = 4\pi \K \sigma $, the field just
outside the conductor must be normal to the surface and the charge density
must be $\sigma$.  Note that $\vEtot$ not necessarily the original applied field
$\vEext$ but is the sum of the fields due to the charges in the conductor
and the external field. % In the case show in figures 3 and 4, the charges
%yield no net field outside the conductor so $\vEtot = \vEext$ but for a 
%conducting sphere the situation is messier.



\pffig{figures/figure_97_04_3.eps}{4 in}{Drawing of a conductor that has just had an  electric field applied to it. Negative charges are experiencing a force opposite to the applied electric field.  }
{Forces on the electrons just after a field is applied
to a conductor}{fig3}


\pffig{figures/figure_97_04_4.eps}{4 in}{Drawing of a rectangular conductor at equilibrium in an electric field with all the charge localized on the surface. Inside the conductor, the electric field is now zero as it has been cancelled by the surface charges. The surface charges are $\pm E_{ext}/(4 \pi k)$. }{The same conductor once things have settled down;
there is a net negative charge on one surface and a net positive charge on the
other due to the motion of the electrons to balance the field.}{fig4}
\end{document}

\pffig{figure_97_04_5.eps}{4 in}{The electric field near a spherical conductor
immersed in a constant electric field $\vEext$.}{fig5}

The presence of an uncharged isolated conductor will generally not change
$\vEtot$ far way because any effect due to the charges will fall off
quickly with distance, especially as the net charge is zero.  But close by
it can have a large effect due to the requirement that $\vEtot$ be normal 
to the surface.

\pffig{figure_97_04_6.eps}{4 in}{The electric field lines for a conducting 
sphere in the field of a point charge $Q$.}{fig6}

It is very tricky to do superposition in cases with conductors as one gets

$$\vE(\vr) \gt \sigma({\vr}) \gt \vE^{\prime}(\vr)
 \gt \sigma^{\prime}(\vr) \gt ....$$

until equilibrium is reached.

As you generally don't know what $\sigma(\vr)$ is on the surface of the conductor,
you can't use 
$$\vE(\vr) = \int \oneovereps \sigma(\rprimev){(\vr - \rprimev)\over
|\vr - \rprimev |^3} dA^{\rprime}$$ 
to find $E$.

Next we will look at Potential Methods which make this problem simpler.
In the real world one generally uses numerical methods but they converge must faster
if one starts with a decent understanding of the underlying physics.

\ignore{
\section{Break for an in class exercise}

A line of charge with 1 C/m is encased in an uncharged cylindrical conducting case with inner radius 10 cm  and outer radius 12 cm.  
What is the charge on the inner surface of the case? 
Reminder - the field due to a line of charge is:
$$\V E(\V r) = \frac{1}{2\pi\epsilon_0} \frac{\lambda}{|r|}\rhat$$

Hint: How would you make the field inside the conductor 0 by moving charges inside the conductor. 

More to think about: 
What is the charge density on the inner surface? 
The field due to a surface charge (very close to the surface if it is curved) is 
$$\V E \sim \frac{1}{\epsilon_0} \sigma\nhat$$

%If you can do this, you can do part a of 2.38.
}


\section{Introduction of Work and Potentials}
So far we have used forces and fields but in many situations, energy
methods work much better.  One can get rid of those pesky vectors.

So what is the work done in moving a charge in an electric field?

$$W_{a\gt b}  = \int_a^b \V{F}\cdot d\V{x}$$

Here $\V F$ is the force acting on the charge as it moves ($q E$).

The change in potential energy $\Delta U$ is related to the work done in  going from 
point $a$ to point $b$.

$$\Delta U = -W = -\int_a^b \V{F}\cdot d\V{x} = - q \int_a^b \V{E}\cdot d\V{x}$$

Where $q$ is the charge we are moving and $\V E$ is the electric field which
causes the force which does the work.

\begin{tcolorbox}
Warning!  Jackson uses the notation $W_i$ to mean the potential energy, not its opposite.
\end{tcolorbox}

\section{The electrostatic potential}

Since we now know we can define work for static electric fields generated by Coulomb's law, we
can define an "electrostatic potential" which works for all test charges. This electrostatic potential is
related to work just as the electric field is related to force.

We can define the electrostatic potential difference $\Delta V$ or
 voltage difference
$\Delta V$ as follows.

$$\Delta V = \Delta U/q =  -\int_a^b \V{E}\cdot d\V{x}$$

Remember that the change in potential energy $\Delta U$ had a factor of $q$ in it while the change in electrostatic potential $\Delta V$ does not. 

$\Delta V$ is a property of the electrical configuration (and of the 
endpoints of the line) but not of the test charge.  It can be used to 
solve the general problem, all you need to get the work done in a situation
is use the expression:

\pffig{figure_97_05_1.eps}{2 in}
{Work done in going from point $\V{a}$ to point $\V{b}$.}{fig1}

$$W = - \Delta U = - q \Delta  V$$

If one fixes $ V=  V_0$ at a reference point $\V{r}_0$,
 you can then define $ V(\V{r}) = \Delta  V(\vr_0 \gt \vr) +  V_0$.


The units of electrostatic potential are Joules/Coulomb or Volts.
This is the same Volts as in  AAA batteries.
There is an electrostatic  potential difference of 1.5 Volts between the 
ends of the battery.





\section{Relation between the potential $ V$ and the electric field $\vE$}

If $$ V(\vr) = - \int_{\vr_0}^{\vr} \vE\cdot d\vr +  V_0(\vr_0)$$
then if we differentiate:

$$\gradient  V(\vr) = -\vE(\vr)$$

You can think of the potential $ V$ as having hills and valleys, the electric
field points down the hill.

The good news:  You can solve many problems in terms of $ V$ a scalar
quantity and then take the gradient at the end to get $\vE$.

A warning: Just like potential energy, the electrostatic potential has no
absolute setting.  Everything has to be compared to a reference point 
$ V(\vr_0)$.   

In undergrad $E\&M$ $V$ is mostly simple and goes to $0$ at infinity.  In this class, we learn to deal with cases where the boundary conditions are more complicated. 

%Normally we define $ V = 0$ at infinity but this is not
%always the best thing to do.


\section{Example:  The potential due to an isolated point charge}

Take a point charge $Q$ at the origin. Then 
$$E(\vr) =  \K Q {\rhat\over r^2}$$

The change in potential in going from point $a$ to point $b$ will be

$$\Delta V(\vr_a \gt \vr_b) = - \int_a^b \vE({\vr})\cdot d \vr$$

As the electric field is radial, only the radial components contribute and the 
integral simplifies to:

$$\Delta V(\vr_a \gt \vr_b) = - \int_{r_a}^{r_b} E(r) dr = \K Q (
{1\over {r_b}} - {1\over {r_a}})$$

$\Delta V$ only depends on the radii, not on any other component of $\vr$.


Say we define $ V_0 = 0$ at infinity, then we can define the function
$$ V(\vr) =  V(r) = - \int_{\infty}^r \K {Q\over r^2} dr +  V(\infty) 
= \K {Q\over r}$$

This is the contribution to the potential due to a single charge at the origin.

If the charge were at $\rprime$ then

$$ V(\vr) = \K {Q\over{|\vr - \rprimev|}}+V(\infty).$$

If there is a charge density $\rho(\rprimev)$ then

$$ V(\vr) = \K \int{\rho(\rprimev)\over{|\vr - \rprimev|}}d V^{\prime}+V(\infty).$$

Here I have left the $V(\infty)$ just to remind you that, in some cases you may
have a different reference point.

%{\LARGE \bf Lecture 5: Work, Energy and the Electrostatic Potential}

\ignore{
\section{Example 2: Potential due to a ring of charge}

A ring of radius $a$ lies in the $x-y$ plane and has a charge $Q$ on it.
What are the electrostatic potential and electric field along the $z-axis$?

First, the length of the ring is $2 \pi a$ so the charge density $\lambda
= Q/L = Q/(2\pi a)$.

Use the simple expression for the potential:

$$V(0,0,z) = \int \K {\lambda d l^{\prime} \over {|\vr - \rprimev|}}$$

All points on the loop lie at radius $a$ and at $\phi^{\prime}$ ranging from 
$0$ to $2 \pi$ in cylindrical coordinates. So

$$dl^{\prime} = a d\phi^{\prime} \hbox{\ and}$$

$$|\vr - \rprimev| = \sqrt{a^2 + z^2}$$

\pffig{figure_97_05_2.eps}{2 in}{A ring of charge of radius $a$ in the $x-y$ plane.
What are the potential and field along the $z$-axis?}{fig2} 

So the integral becomes:

$$V(0,0,z) = \int_0^{2\pi} \K {\lambda a d \phi^{\prime} \over \sqrt{a^2 + z^2}} =
k {\lambda a 2\pi\over\sqrt{a^2 + z^2}} = \K {Q\over\sqrt{a^2 + z^2}}$$

The electric field can be found by first noticing that, on the axis,
$\vE$ can only have components in the $z$ direction because of symmetry.

That means that $$\vE = -\gradient V(\V r) = -\zhat \delz{V} = 
\zhat  \K { z Q\over(a^2+z^2)^{3\over 2}}$$


\section{Extra section: Review of the Gradient Operator}

%\def\gradient{\V{\grad}}
$$\gradient = \xhat \delx{\ } + \yhat \dely{\  } + \zhat \delz{\ }$$

\subsection{Properties of the gradient}


 The gradient is linear:

$$\gradient (a f(\vr)) = a \gradient f(\vr) \hbox{ \ if $a$ is a constant}$$

$$\gradient(f(\vr) + g(\vr) ) = \gradient f(\vr) + \gradient g(\vr)$$

\subsection{Example}

A useful example is the gradient of a function which only depends on the length
of the vector $\vr$ not on its direction.  If $|\vr| = r = \sqrt{x^2 + y^2 + z^2}$ and the function
$f$ depends on $r$, then:

$$\gradient f(r) = \xhat {df\over dr}\delx{r} + 
                   \yhat {df\over dr}\dely{r} + 
                   \zhat {df\over dr} \delz{r} $$
$$\ \ \  \ = {df\over dr}({x\over r} 
\xhat + {y\over r} \yhat + {z\over r} \zhat)$$
$$\ \ \ \ = {df\over dr} \rhat$$



\subsection{What is the Curl operator?}

$$\del\cross \V V  = \xhat ({\partial V_y\over\partial z}-{\partial V_z\over\partial y}) +
\yhat({\partial V_z\over\partial x}-{\partial V_x\over\partial z})+
\zhat({\partial V_x\over\partial y}-{\partial V_y\over\partial x})$$

It is also linear

$$\del\cross(a \V F(\vr)) = a \del\cross \V F(\vr) \hbox{ \ if $a$ is a constant}$$

$$\del\cross(\V F (\vr) + \V G(\vr) ) = \del\cross \V F(\vr) + \del\cross \V G (\vr)$$
}
\section{Can you define "work" for electrostatic fields?}

To define the work in going from point a to point be, you need to know that the integral 
of the work done doesn't depend on the path taken - for an arbitrary vector field it can and we will find that once we have charges moving around (ie. ${\partial \V E\over \partial t} \ne 0$) there may
be a path dependence.

But for now let's assume our charges are stationary and Coulomb's law holds.

 Stokes' Theorem tells us that:

$$\int_S (\del\cross\V F)\cdot d\V A = \oint \V F \cdot d\V \cal l$$

Now let's try two different paths between $a$ and $b$.

Then let $$W_1 =  \int_a^b q \V E\cdot d\V \cal l_1$$ be the work done along  path 1

and $$W_2 = \int_a^b  q \V E\cdot d\V \cal l_2$$ be the work done on path 2.

$$W_1 - W_2 =  \oint_{a\gt b\gt a} q \V E\cdot d\V \cal l$$ is the integral around the loop going from $a$ to $b$ along path 1 and then back from $b$ to $a$ along path 2.
We would obviously like $W_1$ and $W_2$ to be the same so that we can define the work.  That
requires that  $$W_1-W_2 = \oint  q \V E\cdot d\V \cal l= 0$$
Stokes' Theorem tells us that
$$- \oint  q \V  E\cdot d\V \cal l  = q \int_S(\del\cross\V E) \cdot d\V A$$ when you integrate across the
surface $S$ which is surrounded by the path from $a\gt b\gt a$.

So if $\del\cross\V E = 0$ everywhere on the surface, we know that $W_1 = W_2$.

So what is $\del\cross\V E$ for a static electric field? A static field is just the sum of fields due to individual point charges. If we can show that $\del\cross \V E = 0$ at all points for a single point charge, then it is true by superposition and linearity for all of them.

For $$\V E(\V r) = \K Q {\rhat\over r^2}$$

you can (with a bit of work) show that $\del\cross \V E(\V r) = (0,0,0)$, which means that a static field has no curl, so the integral of the work done in a static field is the same for any path.


 \end{document}