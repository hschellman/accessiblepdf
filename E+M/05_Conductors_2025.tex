%!TEX TS-program = latex

%%% Define flags for accessibility


\newif\ifaccess  % test accessibility, if not true, accessibility tools like tagpdf are not enabled. 
\newif\ifuatwo   % enable PDF 2.0/UA-2 - alternative is PDF 1.7/UA-1 
\newif\iflargeprint

%%% Set flags for different accessibility levels
 
\accesstrue
\uatwofalse  % This toggles between PDF 1.7 and PDF 2.0
\largeprintfalse % This toggles largeprint options

\input accpreamble.tex

\iflargeprint
\documentclass[17pt]{extarticle}
\else
\documentclass[12pt]{extarticle}
\fi

\input PH631_layout2
\begin{document}

\input extra_macros
\input PH631_logo\setcounter{part}{5}



\bigskip

{\LARGE \bf Conductors and the Electrostatic Potentlal}

\begin{tcolorbox}
We're going to do Jackson sections 1.1-1.5 and then do 1.11 on energy and capacitance before looping back to boundary conditions and Greens' functions (1.7-1.10) after we solve some potential problems using the method of images. 
\end{tcolorbox}

\section{Conductors}

So far we have only used fixed charge distributions.
Now we will introduce the 'ideal' conductor which has the property that
the charges within it can move freely under the influence of electrical forces.

Say we take a chunk of conducting material and immerse it in a uniform
electric field?  All of the free electrons will experience a force and move
until there is no more force on them.  That force will be $q\vE_{\rm TOT}$ where
the total field is the sum of the external field and fields due to the electrons
and protons themselves. The electrons will move so that their field cancels
the external field.  If $F= 0 $ inside the conductor, $\vE_{\rm TOT}$ must be
as well once the charges stop moving.
\bigskip

\centerline{AT EQUILIBRIUM $\vE = 0$ INSIDE A PERFECT CONDUCTOR}
\bigskip

\subsection{Fields near the surface of the conductor}

As the field is identically zero inside the conductor and the field has to 
satisfy the condition $\Delta \vE_{\rm tot} = 4\pi \K \sigma $, the field just
outside the conductor must be normal to the surface and the charge density
must be $\sigma$.  Note that $\vEtot$ not necessarily the original applied field
$\vEext$ but is the sum of the fields due to the charges in the conductor
and the external field. % In the case show in figures 3 and 4, the charges
%yield no net field outside the conductor so $\vEtot = \vEext$ but for a 
%conducting sphere the situation is messier.



\pffig{figures/figure_97_04_3.eps}{4 in}{Drawing of a conductor that has just had an  electric field applied to it. Negative charges are experiencing a force opposite to the applied electric field.  }
{Forces on the electrons just after a field is applied
to a conductor}{fig3}


\pffig{figures/figure_97_04_4.eps}{4 in}{Drawing of a rectangular conductor at equilibrium in an electric field with all the charge localized on the surface. Inside the conductor, the electric field is now zero as it has been cancelled by the surface charges. The surface charges are $\pm E_{ext}/(4 \pi k)$. }{The same conductor once things have settled down;
there is a net negative charge on one surface and a net positive charge on the
other due to the motion of the electrons to balance the field.}{fig4}

\end{document}
